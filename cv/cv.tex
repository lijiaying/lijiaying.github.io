\documentclass[a4paper,10pt]{article}

%A Few Useful Packages
\usepackage{marvosym}
\usepackage{fontspec} 					%for loading fonts
\usepackage{xunicode,xltxtra,url,parskip} 	%other packages for formatting
\RequirePackage{color,graphicx}
\usepackage[usenames,dvipsnames]{xcolor}
\usepackage[big]{layaureo} 				%better formatting of the A4 page
% an alternative to Layaureo can be ** \usepackage{fullpage} **
\usepackage{supertabular} 				%for Grades
\usepackage{titlesec}					%custom \section

%Setup hyperref package, and colours for links
\usepackage{hyperref}
\definecolor{linkcolour}{rgb}{0,0.2,0.6}
\hypersetup{colorlinks,breaklinks,urlcolor=linkcolour, linkcolor=linkcolour}

%FONTS
\defaultfontfeatures{Mapping=tex-text}
% \setmainfont[SmallCapsFont = Fontin SmallCaps]{Fontin}
%%% modified for Karol Kozioł for ShareLaTeX use
% \setmainfont[
 %SmallCapsFont = Fontin-SmallCaps.otf,
 %BoldFont = Fontin-Bold.otf,
 %ItalicFont = Fontin-Italic.otf
 %]
%{Fontin.otf}
%%%

%CV Sections inspired by: 
%http://stefano.italians.nl/archives/26
\titleformat{\section}{\Large\scshape\raggedright}{}{0em}{}[\titlerule]
\titlespacing{\section}{0pt}{3pt}{3pt}
%Tweak a bit the top margin
%\addtolength{\voffset}{-1.3cm}

%Italian hyphenation for the word: ''corporations''
\hyphenation{im-pre-se}

%-------------WATERMARK TEST [**not part of a CV**]---------------
\usepackage[absolute]{textpos}

\setlength{\TPHorizModule}{20mm}
\setlength{\TPVertModule}{\TPHorizModule}
\textblockorigin{2mm}{0.35\paperheight}
\setlength{\parindent}{0pt}

%--------------------BEGIN DOCUMENT----------------------
\begin{document}


\pagestyle{empty} % non-numbered pages

\font\fb=''[cmr10]'' %for use with \LaTeX command

%--------------------TITLE-------------
\par{\centering
		{\Huge Jiaying \textsc{Li}
	}
\bigskip
\par}

%--------------------SECTIONS-----------------------------------
%Section: Personal Data
% \section{Personal Data}

\begin{tabular}{rl}
%   \textsc{Place and Date of Birth:} & Someplace, Italy  | dd Month 1912 \\
    \textsc{Address:}   & 1.304, 8 Somapah Road, 487372, Singapore\\
    \textsc{Phone:}     & +65 9132 7556\\
    \textsc{email:}     & \href{mailto:lijiaying1989@gmail.com}{lijiaying1989@gmail.com}\\
	\textsc{website:}   &  \href{http://jiaying.li}{http://jiaying.li}
\end{tabular}

%Section: Education
\section{Education}
\begin{tabular}{rl}	
 \textsc{Sep.} 2013 - \textsc{Jun.} 2018  & Ph.D. in {Information Systems of Technology and Design}, \\
& \textbf{Singapore University of Technology and Design}, Singapore\\
% & Research area: Software Engineering, Program Analysis\\
% & Thesis: ``Sublinear and Locally Sublinear Prices'' \\
& \small Advisor: Assoc. Prof. Jun Sun\\

\textsc{Sep.} 2011 - \textsc{Jun.} 2013& Study in 
% Institute of Computing Technology,
{Institute of Computing Technology}\\
& \textbf{Chinese Academy of Sciences}, Beijing, China\\
& \small Advisors: Lu Xu, Jianfeng Zhan\\
% & \textsc{Sep.} 2011 - \textsc{Nov.} 2012 \\
% & ~~~~~~~~\textsc{Data Storage and Management Technology Research Center} \\
% & \small Advisor: Lu Xv\\
% & \textsc{Dec.} 2012 - \textsc{Jun.} 2013 \\
% & ~~~~~~~~\textsc{Advanced Computer Research Center} \\
% & \small Advisor: Jianfeng Zhan \\
% &\normalsize \textsc{Gpa}: 3.72/4.3; \textsc{Rank}: 5/44 \\


\textsc{Sep} 2007 - \textsc{Aug.} 2011& B.Eng. in {College of Software} \\
& \textbf{Nankai University}, Tianjin, China\\
&\normalsize \textsc{Gpa}: 3.9/4.0; \textsc{Rank}: 3/134 
% & \hyperlink{grds_cleli}{\hfill| \footnotesize Detailed List of Exams}\\&\\
\end{tabular}

\section{Research Interests}
Software engineering, program analysis, machine learning

%Section: Project Experience
\section{Research Experience}
\begin{tabular}{rp{11cm}}
\textsc{Sep.} 2016 - \textsc{Aug.} 2017 & \textsc{Zimu:} \footnotesize{Complex invariant learning framework based on program structures}\\
%& \textsc{Singapore University of Technology and Design, Singapore} \\
%& Advisor: Jun Sun, SUTD\\
&\footnotesize{
\begin{itemize}
	\item Design an algorithm to partition program based on program structuren and learn disjunctive invariants based on the resulting partitions;
	\item Design a heuristic algorithm to search for good program partitionings that lead to simpler loop-invariant by exploring program structure adaptively.
%	\item Evaluate the performance with a comprehensive set of benchmarks. The results show that our approach is more effective in learning disjunctive loop-invariants compared with existing approaches.
\end{itemize}
} \\


\textsc{Oct.} 2016 - \textsc{Apr.} 2017 & \textsc{pta-Learn:} \footnotesize{Parametric timed automata verification with active learning}\\
%& \textsc{Singapore University of Technology and Design, Singapore} \\
%& Advisor: Jun Sun, SUTD\\
&\footnotesize{
\begin{itemize}
	\item Enhance the scalability of existing model checkers for PTA by adopting machine learning techniques;
	\item Form conjectures on the constraint based on sampling and classification; 
%Firstly, we generate random parameter values and construct the corresponding non-parametric timed automata. Next, we verify the timed automata using existing model checker (i.e., \textsc{Uppaal}).
%Based on the checking results, we form conjectures on the constraint through machine learning, which can be subsequently checked using existing model checkers for PTA (i.e., \textsf{IMITATOR}).
	\item Actively seek out informative parameter values and check the corresponding timed automata so that we converge to an accurate conjecture quickly.
%	\item Evaluate it on benchmark systems.
\end{itemize}
} \\


\textsc{Dec.} 2015 - \textsc{Jun.} 2016 & \textsc{Zilu:}   \footnotesize{Invariant generation framework based on classification and selective sampling}\\
%& \textsc{Singapore University of Technology and Design, Singapore} \\
%& \small{Advisor: Jun Sun, SUTD}\\
&\footnotesize{
\begin{itemize}
	\item Build an invariant learning framework based on program variable valuations in run-time and classification algorithms;
	\item Propose an active learning technique, known as selective sampling, to overcome the limitation of random sampling;
%	\item Propose to generate disjunctive invariants through {\em path-sensitive} learning
	\item Actively seek out informative parameter values and check the corresponding timed automata so that we converge to an accurate conjecture quickly;
%	\item Evaluate it on benchmark systems and compare with state-of-the-art tools like Interproc, CPAChecker, InvGen and BLAST.
\end{itemize}
}\\

%\textsc{June} 2015 - \textsc{Dec.} 2015 & \textsc{FIIF} \\
%& \textsc{Singapore University of Technology and Design, Singapore} \\
%& Advisor: Jun Sun, SUTD\\
%&\footnotesize{Design an mobile imaging system that detect and conduct early diagnose on skin mole of being melanoma or not. Responsible for design the skin mole detection model, test on a skin mole image dataset with %different resolutions and lighting conditions. Implement the whole system on mobile device.} \\
%\\

\textsc{Oct.} 2010 - \textsc{June.} 2011 & \textsc{An Operating System Kernel} \\
%& \textsc{Nankai University, China} \\
%& Advisor: Yaoguo Li, Nankai University\\
&\footnotesize{
\begin{itemize}
	\item Design a micro-kernel of operating systems based on the book `Orange S' and Linux kernel source code version 0.11;
	\item Implement most of the basic kernel functions, including process management, I/O, interprococess communication, file system, memory management.
\end{itemize}
} \\
\end{tabular}







%Section: Working
\section{Working Experience}
\begin{tabular}{rl}	
	\textsc{Jul.} 2013 - \textsc{Sep.} 2013  & Research assistant in Information Systems of Technology and Design,\\
		 									 & Singapore Universify of Technology and Design, Singapore \\
	\textsc{Dec.} 2012 - \textsc{Jun.} 2013	 & Research assistant in Advanced Computer Research Center,\\
		 									 & Institute of Computing Technology, Chinese Academy of Sciences, Beijing, China \\
	\textsc{Mar.} 2012 - \textsc{Nov.} 2012  & Research assistant in Data Storage and Management Technology Research Center, \\
											 & Institute of Computing Technology, Chinese Academy of Sciences, Beijing, China \\
	\textsc{Oct.} 2011 - \textsc{Jan.} 2012  & Developer in Trend Media Corporation Limited., Beijing, China\\
	\textsc{Oct.} 2009 - \textsc{Jan.} 2010  & Developer in Sun Micro systems, Inc., Beijing, China\\
\end{tabular}






%Section: Scholarships and additional info
\section{Scholarships}
\begin{tabular}{rl}
\textsc{Sep.} 2013 - \textsc{Aug.} 2018 & SUTD President’s Graduate Fellowship \\
\textsc{Oct.} 2010 - \textsc{Jun.} 2011 & First Prize National Fellowship \\
\textsc{Oct.} 2009 - \textsc{Jun.} 2010 & National Scholarship \\
\textsc{Oct.} 2008 - \textsc{Jun.} 2009 & National Endeavor Scholarship \\
\textsc{Oct.} 2007 - \textsc{Jun.} 2008 & Second Prize National Fellowship \\
\end{tabular}



%Section: Languages
\section{Languages}
\begin{tabular}{rl}
	\textsc{Chinese:}&Mothertongue\\
	\textsc{English:}&Fluent\\
% \textsc{French:}&Basic Knowledge\\
\end{tabular}






\section{Publication list}

\textbf{Jiaying Li}, Jun Sun, Learning Disjunctive Invariants based on Loop Structures. {\sl In the 40th International Conference on Software Engineering (ICSE'18),} Gothenburg, Sweden, 2018. (under review)

\textbf{Jiaying Li}, Jun Sun, Bo Gao and ~ ́Etienne Andre ́, Classification-based Parameter Synthesis for Parametric Timed Automata. {\sl In the 19th International Conference on Formal Engineering Methods (ICFEM'17),} Xi'an, China, 2017.

\textbf{Jiaying Li}, Jun Sun, Li Li, Quang Loc Le and Shang-Wei Lin, Automatic Loop-invariant Generation and Refinement through Selective Sampling. {\sl In the 32nd IEEE/ACM International Conference on Automated Software Engineering (ASE'17),} Illinois, USA, 2017.

Truong Khanh Nguyen,Tian Huat Tan, Jun Sun, \textbf{Jiaying Li}, Yang Liu, Manman Chen, Jin Song Dong, Scaling BDD-based Timed Verification with Simulation Reduction, {\sl The 19th International Conference on Formal Engineering Methods (ICFEM'16),} Tokyo, Japan, 2016. 

\textbf{Jiaying Li}, An Invariant Inference Framework using Active Learning and SVMs, {\sl In the 20nd International Conference on Engineering of Complex Computer Systems (ICECCS'15),} Doctorial Symposium, Gold Coast, Austrilia, 2015.








\section{Services}
\begin{tabular}{ll}
Volunteer:&FM 2014\\
Presentation: &ICECCS 2015, ICFEM 2016\\
Teaching assistant:&Machine learning (undergraduate, ISTD, SUTD, 2014)\\
&Elements of software construction (undergraduate, ISTD, SUTD, 2014)
\end{tabular}





\section{Skills}
\begin{tabular}{ll}
Programming:& Professional in C, C++, bash shell \\%, proficient in Python, Java\\
Operating Systems: & Professional in Linux %, proficient in Windows and Mac
% Develop experience:& Android, Html
\end{tabular}





% \section{Interests and Activities}
% Technology, Open-Source, Programming, Marathon, Football, Diving, Swimming, Tennis

% \newpage

\end{document}